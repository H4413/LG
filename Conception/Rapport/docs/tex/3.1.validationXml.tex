\subsection{Validation sémantique d'un document XML}

La validation sémantique d'un document XML consiste à vérifier : 

\begin{itemize}
\item que toutes les balises utilisées dans le document sont définies dans
la DTD associée
\item que les noeuds fils de chaque balise sont ordonnés selon la
définition de la balise mère dans la DTD
\end{itemize}

La procédure de validation d'un document consiste à valider son élément
racine.\\
La classe XMLElement est dotée d'une fonction ValidateElement :

\begin{algorithmic}
\STATE Recherche de l'élément DTD associé à l'élément XML

\IF {Élément non trouvé}
\RETURN Faux
\ENDIF

\STATE result $\gets$ Validation de l'élément DTD

\IF{result}

\FORALL{Élément XML fils de l'élément courant e}
\STATE result $\gets$ result \AND e.Validate
\ENDFOR

\RETURN result

\ELSE
\RETURN Faux
\ENDIF
\end{algorithmic}


\subsection{Algorithme de validation d'un élément DTD}

L'algorithme de validation d'un élément DTD valide l'objet DTDContentSpec
associé, en se basant sur la liste des noeuds fils de l'élément XML en
cours de validation. La fonction prend également en entrée un itérateur sur
ce vecteur de noeuds.

\subsubsection{DTDAny, DTDEmpty}

La fonction Validate de DTDAny renvoie toujours vraie. Celle de DTDEmpty
renvoie vraie si le vecteur d'éléments XML passé en argument est vide.

\subsubsection{DTDName}

La fonction renvoie vraie si l'élément XML du vecteur sur lequel pointe
l'itérateur a le même nom que l'élément DTD appelant la fonction Validate.
Si l'élément XML est de type XMLContent (chaîne de caractère), la fonction
renvoie vrai si l'élément DTD a pour nom #PCDATA.

\begin{algorithmic}
\STATE input arguments : nodeVector, nodeIt
\STATE bool result
\IF{nodeIt = nodeVector.end}
\STATE result $\gets$ faux
\ELSE
\STATE result $\gets$ ( name = nodeIt.getName )
\ENDIF

\IF{result}
\STATE nodeIt $\gets$ nodeIt + 1
\ENDIF

\IF{mark = M\_Q}
    \IF{result}
    \STATE IsValidated( nodeVector, nodeIt )
    \ENDIF
    \RETURN vrai
\ELSIF{mark = M\_PLUS}
    \IF{result}
    \STATE IsValidated( nodeVector, nodeIt )
    \ENDIF
    \RETURN result
\ELSIF{mark = M\_AST}
    \IF{result}
    \STATE IsValidated( nodeVector, nodeIt )
    \ENDIF
    \RETURN vrai
\ENDIF

\RETURN result

\end{algorithmic}


\subsubsection{DTDSequence}

La fonction fonctionne sur le même principe que celle de DTDName, à deux
différences près : 
\begin{itemize}
\item le critère de validation est différent : si tous les éléments DTD
contenus dans la séquence sont validés, la fonction renvoie vrai également.
\item le nodeIt n'est pas incrémenté en cas de validation ; en revanche on
en fait une sauvegarde en début de fonction afin de le restaurer si la
séquence n'est pas validée. Il s'agit d'une sorte de "point de reprise" (si
la séquence est une possibilité d'un choix, la fonction IsValidated de
l'élément choix peut essayer une autre possibilité en repartant du même
point).
\end{itemize}

\subsubsection{DTDChoice}

La fonction de validation est similaire à celle de DTDSequence sauf pour ce
qui est du critère de validation : si un des éléments DTD contenus dans le choix est validé, la fonction
renvoie vrai.

