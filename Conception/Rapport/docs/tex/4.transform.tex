\section{L'algorithme de transformation}
	\subsection {Fonctionnement}
	L'algorithme de transforamation a été implementé avec la 
	technique algorithmique \og Diviser pour régner \fg. Nous avons
	encapsulé toutes les fonctions nécessaires à la transformation 
	dans la classe \textit {XslTransform}. 
	L'algorithme suit les étapes suivantes: \\

	\begin{itemize}
	\item Parcours de l'arbre XML à partir de l'appel à la fonction 
		\textbf{nextNode}.
	\item Recherche d'un template applicable en utilisant la fonction 
		\textbf{searchForTemplate}.
	\item Si template applicable, on affiche les nodes fils du 
		template trouvé (\textbf{printTemplate}).
	\item Si un des nodes fils est \textit{apply-templates}, on
		appelle la méthode \textbf{applyTemplates} récursivement.
	\item Si un des nodes fils est \textit{value-of}, on
		affiche l'élément XML courant en appelant la fonction
		\textbf{valueOf}. \\
	\end{itemize}	
	
	Nous avons quelques variables communs à l'ensemble des méthodes:
	
	\begin{itemize}
		\item[- currentNode] Pointeur vers l'élément courant
		\item[- xmlRoot] Pointeur vers l'élément root du fichier XML
		\item[- xslRoot] Pointeur vers l'élément root du fichier XSL
		\item[- parents] Profondeur de l'arbre, cela
		correspond à la sauvegarde temporaire des noeuds parents. \\
	\end{itemize}
	
	
	Pour initialiser l'algorithme, on récupère les éléments roots
	des deux arbres (XML et XSL), on pointe l'élément courant vers
	l'élément root de l'arbre XML et on appelle la méthode 
	\textit{applyTemplate(true)}. \\ \\
	
	\subsection {Listing}
	\lstset{language=c++} 
	\lstset{commentstyle=\textit} 
		
	\subsubsection {Apply-Templates}
	\begin{lstlisting} 
void XslTransform::applyTemplates(bool isRoot)
{
	XmlNode * templateNode;	

	if (!isRoot)
		nextNode();
	
	do {
		templateNode = searchForTemplate();
		if (templateNode)
			printTemplate(templateNode);
	} while(nextNode());
}
	\end{lstlisting} 

	\subsubsection {Prochain noeud}
	\begin{lstlisting} 
bool XslTransform::nextNode()
{
	if (!currentNode)
		return false;
		
	if (currentNode->HasChild())
	{
		parents.push(currentNode);
		currentNode = currentNode->FirstChild();
		return true;
	}
		
	while (!parents.empty())
	{
		currentNode = parents.top()->NextChild();
		
		if (currentNode)
			return true;
		
		parents.pop();
	}
	
	currentNode = NULL;
	return false;
}
	\end{lstlisting} 

	\subsubsection {Rechercher template}
	\begin{lstlisting} 
XmlNode * XslTransform::searchForTemplate()
{	
	if (currentNode->IsContent())
		return NULL;
		
	XmlNode * xslChild = xslRoot->FirstChild();
	while (xslChild != NULL)
	{
		if (xslChild->HasAttributes())
		{
			if (xslChild->GetAttribute("match")->Value 
				== currentNode->NodeName())
				return xslChild;
			else if (xslChild->GetAttribute("match")->Value == "/"
					&& xmlRoot == currentNode)
				return xslChild;
		}
		xslChild = xslRoot->NextChild();
	}
	return NULL;
}
	\end{lstlisting} 

	\subsubsection {Afficher template}
	\begin{lstlisting} 
void XslTransform::printTemplate(XmlNode * templateNode)
{
	XmlNode * xslNode = templateNode->FirstChild();
	
	while (xslNode != NULL)
	{		
		if (xslNode->NodeName() == "apply-templates")
		{
			applyTemplates();
		}
		else if (xslNode->NodeName() == "value-of")
		{
			valueOf(currentNode);
		}
		else if (xslNode->IsContent())
		{
			*out << ((XmlContent*)xslNode)->Content();
		} 
		else
		{
			if (!xslNode->HasChild())
			{
				*out << "<" << xslNode->NodeName() << "/>" << endl;	
			} else {
				begin(xslNode);
				printTemplate(xslNode);
				end(xslNode);			
			}
		}
		xslNode = templateNode->NextChild();	
	}
}
	\end{lstlisting} 
	
	\subsubsection {Value-Of}
	\begin{lstlisting} 
void XslTransform::valueOf(XmlNode * node)
{
	if (node->IsContent())
		*out << " " << ((XmlContent*)node)->Content();
	
	if (node->HasChild())
	{
		NodeList::const_iterator it;
		for ( it = node->Children()->begin(); 
				it != node->Children()->end(); ++it )
		{
			valueOf(*it);
		}
	}
}
	\end{lstlisting} 
